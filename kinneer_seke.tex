%!TEX root=kapfhammer_gatorday2015_presentation.tex
% mainfile: kapfhammer_gatorday2015_presentation.tex

\documentclass[hyperref]{beamer}
% \includeonlyframes{current}

\usecolortheme[accent=blue,dark]{solarized}

\beamertemplatetransparentcovered

\usepackage[utf8]{inputenc}
\usepackage{merriweather}
\usepackage{moresize}
\usepackage{anyfontsize}
\usepackage{xcolor}
\usepackage{graphicx}

\usepackage{pgfplots}
\pgfplotsset{compat=1.9}
\usepgfplotslibrary{colormaps,external}

\usepackage{minted}
\usemintedstyle{solarized}

\usepackage{pifont}
\newcommand{\cmark}{{\color{solarizedGreen}\ding{51}}}
\newcommand{\xmark}{{\color{solarizedOrange}{\ding{55}}}}
\newcommand{\cmarkhide}{{\color{kapfhammerDarkGrey}\ding{51}}}
\newcommand{\xmarkhide}{{\color{kapfhammerDarkGrey}{\ding{55}}}}

\usepackage{tikz}
\usetikzlibrary{positioning,shadows,arrows,shapes,calc,backgrounds}

\setbeamercolor{background canvas}{bg=kapfhammerDarkGrey}

\setbeamertemplate{section in toc shaded}[default][65]
\setbeamertemplate{subsection in toc shaded}[default][65]

\setbeamertemplate{navigation symbols}{}

\setbeamerfont{title}{size=\HUGE,series=\rmfamily,parent=merriweather}
\setbeamerfont{frametitle}{size=\HUGE,series=\rmfamily,parent=merriweather}
\setbeamerfont{framesubtitle}{size=\normalsize,series=\rmfamily,parent=merriweather}
\setbeamerfont{subtitle}{size=\normalsize,series=\bfseries,parent=merriweather}
\setbeamerfont{author}{size=\LARGE,series=\bfseries,parent=merriweather}
\setbeamerfont{institute}{size=\normalsize,series=\bfseries,parent=merriweather}
\setbeamerfont{date}{size=\normalsize,series=\bfseries,parent=merriweather}

\setbeamercolor{title}{fg=solarizedOrange}
\setbeamercolor{subtitle}{fg=solarizedViolet}
\setbeamercolor{frametitle}{fg=solarizedRebase00}
\setbeamercolor{framesubtitle}{fg=solarizedRebase00}
\setbeamercolor{author}{fg=solarizedRebase00}
\setbeamercolor{institute}{fg=solarizedRebase00}
\setbeamercolor{date}{fg=solarizedRebase00}

\addtobeamertemplate{frametitle}{\vskip.1in}{}

\title{lolwut}

% \title{Automatically Evaluating the Efficiency of Search-Based Test Data Generation for Relational Database Schemas}

\subtitle{That title is just excessive}

\author[Kinneer-Kapfhammer]{Cody Kinneer \and Gregory M.\ Kapfhammer}
\institute[Allegheny College]{Department of Computer Science\\ Allegheny College}
\date[Feb 23, 2015]{March 31, 2015}

\begin{document}

\begin{frame}
  \titlepage
\end{frame}

% \begin{frame}
%   \tableofcontents
% \end{frame}

%%%%%%%%%%%%%%%%%%%%%%%%%%%%%
% The Challenges of Software
%%%%%%%%%%%%%%%%%%%%%%%%%%%%%

%intro
\section{Search-based Software Testing}
  \begin{frame}
    \frametitle{Search-based Software Testing}
    \begin{itemize}
      \item EvoSuite
      \item SchemaAnalyst
      \item \in O(?)
    \end{itemize}
  \end{frame}

  %background
  \section{Background}

    \begin{frame}
      \frametitle{Measuring Performance}
      \begin{itemize}
        \item Scalability
        \item O()
        \item Doubling experiment
      \end{itemize}
    \end{frame}

    \begin{frame}
      \frametitle{SchemaAnalyst}
      \begin{itemize}
        \item Criterion
        \item Data generator
        \item Schema
      \end{itemize}
    \end{frame}

    \section{Technique}
      \begin{frame}
        % \begin{figure}
          \frametitle{Method of Approach}
          \centering
          \tikzstyle{proc} = [draw, thick, fill=solarizedViolet, text centered, rounded corners,
    text=solarizedRebase02, draw=solarizedViolet]

\tikzstyle{prochighlight} = [draw, thick, fill=solarizedOrange, text centered, rounded corners,
    text=solarizedRebase02, draw=solarizedOrange]

\tikzstyle{procold} = [draw, thick, fill=solarizedViolet!75, text centered, rounded corners,
    text=solarizedRebase02, draw=solarizedViolet!75]

\tikzstyle{procchanged} = [draw, thick, fill=solarizedViolet!75, text centered, rounded corners,
    text=solarizedRebase02, draw=solarizedViolet!75]

\tikzstyle{prochighlightold} = [draw, thick, fill=solarizedOrange!75, text centered, rounded corners,
    text=solarizedRebase02, draw=solarizedOrange!75]

\tikzstyle{prochighlightchanged} = [draw, thick, fill=solarizedYellow!75, text centered, rounded corners,
    text=solarizedRebase02, draw=solarizedYellow!75]

\tikzstyle{proctest} = [draw, thick, fill=solarizedOrange, text centered, rounded corners,
text=solarizedBase02, draw=solarizedOrange]

\tikzstyle{procnew} = [draw, thick, fill=solarizedGreen, text centered, rounded corners,
    text=solarizedRebase02, draw=solarizedGreen]

\tikzstyle{procyellow} = [draw, thick, fill=solarizedYellow, text centered, rounded corners,
    text=solarizedRebase02, draw=solarizedYellow]

\tikzstyle{procred} = [draw, thick, fill=solarizedRed, text centered, rounded corners,
    text=solarizedRebase02, draw=solarizedRed]

\tikzstyle{io} = [ellipse, draw, thick, fill=solarizedBlue, draw=solarizedBlue, text=solarizedRebase02]

\tikzstyle{iopass} = [ellipse, draw, thick, fill=solarizedGreen, draw=solarizedGreen, text=solarizedRebase02]
\tikzstyle{iofail} = [ellipse, draw, thick, fill=solarizedRed, draw=solarizedRed, text=solarizedRebase02]
\tikzstyle{iohighlight} = [ellipse, draw, thick, fill=solarizedYellow, draw=solarizedYellow,
    text=solarizedRebase02]

\tikzstyle{iofailother} = [ellipse, draw, thick, fill=solarizedYellow, draw=solarizedYellow,
    text=solarizedRebase02]
\tikzstyle{wrongoutput} = [ellipse, draw, thick, fill=solarizedCyan, draw=solarizedCyan, text=solarizedRebase02]

\tikzstyle{special} = [draw, thick, fill=solarizedGreen, text centered, draw=solarizedGreen,
    text=solarizedBase02]
\tikzstyle{specialOrange} = [draw, thick, fill=solarizedOrange, text centered, draw=solarizedOrange,
    text=solarizedBase02]
\tikzstyle{specialGreen} = [draw, thick, fill=solarizedGreen, text centered, draw=solarizedGreen,
    text=solarizedBase02]
\tikzstyle{specialYellow} = [draw, thick, fill=solarizedYellow, text centered, draw=solarizedYellow,
    text=solarizedBase02]

\tikzstyle{pass} = [draw, thick, fill=solarizedGreen, text centered, draw=solarizedGreen, text=solarizedRebase02]
\tikzstyle{fail} = [draw, thick, fill=solarizedRed, text centered, draw=solarizedRed, text=solarizedRebase02]

\tikzstyle{feature} = [draw, thick, fill=solarizedOrange, text centered, text=solarizedRebase02, draw=solarizedOrange]

          \newcommand{\mx}[1]{\mathbf{\bm{#1}}} % Matrix command
\newcommand{\vc}[1]{\mathbf{\bm{#1}}} % Vector command

% Define the layers to draw the diagram
\pgfdeclarelayer{background}
\pgfdeclarelayer{foreground}
\pgfsetlayers{background,main,foreground}

% Define distances for bordering
\def\blockdist{2}
\def\vertdist{2}

\begin{tikzpicture}[thick,scale=0.4, every node/.style={scale=0.4}]
    \node (wa) [proc]  {\textit{SchemaAnalyst}};

    \path (wa.west)+(-\blockdist,1.0*\vertdist) node (asr2)[io] {Coverage Criterion};
    \path (wa.west)+(-\blockdist,0.0*\vertdist) node (dots)[io] {Data Generator};

    \path (wa.west)+(-\blockdist,-1.0*\vertdist) node (asr1) [io] {Database Schema};
    \path (asr1.west)+(-\blockdist,0.0*\vertdist) node (doubler) [proc] {Schema Doubler};

        \path (asr2.west)+(-2.65*\blockdist-.12,-0.0*\vertdist) node (criteriona)
        [io] {Coverage Criterion};
        \path (dots.west)+(-2.65*\blockdist-.12,-0.0*\vertdist) node (dataa) [io] {Data Generator};
    \path (doubler.west)+(-\blockdist,-0.0*\vertdist) node (doublera) [io] {Doubler Choice};
    \path (doubler.west)+(-\blockdist,-1.0*\vertdist) node (schemaa) [io] {Database Schema};

    \path (wa.east)+(\blockdist,0*\vertdist) node (vote) [io] {Test Suite};

    \path [draw, ->] (doubler.east) -- node [above] {}
        (asr1.180);

    \path [draw, ->] (asr1.east) -- node [above] {}
        (wa.200);
    \path [draw, ->] (asr2.east) -- node [above] {}
        (wa.160);
    \path [draw, ->] (dots.east) -- node [above] {}
        (wa.180);
    \path [draw, ->] (wa.east) -- node [above] {}
        (vote.west);

    \path [draw, ->] (doublera.east) -- node [above] {}
        (doubler.west);
    \path [draw, ->] (dataa.east) -- node [above] {}
        (dots.west);
    \path [draw, ->] (criteriona.east) -- node [above] {}
        (asr2.west);
    \path [draw, ->] (schemaa.east) -| node [above] {}
        (doubler.230);

    \path (vote.east)+(\blockdist,0*\vertdist) node (runtime) [io] {Runtime Records};
    \path [draw,->] (vote.east)+(0.3,0) -- node [above]{}
        (runtime.west);
    \path (runtime.east)+(\blockdist,0*\vertdist) node (runtimeo) [io] {Runtime Records};


     \path (vote.east)+(\blockdist,-3.165*\vertdist) node (convalg) [proc] {Converge Algorithm};
     \path [draw,->] (runtime.south) -- node [above]{}
        (convalg.north);
     \path [draw,->] (convalg.west) -| node [above,pos=.25]{Continue Experiment}
        (doubler.290);

        \path [draw,->] (runtime.east) -- node [above]{}
        (runtimeo);


    \path (wa.south) +(0,-1.4) node (asrs) {\textit{SchemaAnalyst} Execution};

    \path (asrs.south) +(0,-1.6) node (singleexp) {\textsc{ExpOse}};

    \begin{pgfonlayer}{background}
        \path (doubler.west |- asr2.north)+(-0.3,0.6) node (a) {};
        \path (wa.south -| runtime.east)+(+0.3,-0.6) node (b) {};
        \path (runtime.east |- singleexp.east)+(+0.3,-0.3) node (c) {};

        \path[fill=kapfhammerDarkGrey,rounded corners, draw=black!50, dashed]
            (a) rectangle (c);
        \path (asr1.north west)+(-0.2,0.2) node (a) {};

    \end{pgfonlayer}

    \begin{pgfonlayer}{background}
        \path (asr2.west |- asr2.north)+(-0.3,0.3) node (a) {};
        \path (wa.south -| wa.east)+(+0.3,-0.3) node (b) {};
        \path (vote.east |- asrs.east)+(+0.3,-0.5) node (c) {};

        \path[fill=kapfhammerDarkGrey,rounded corners, draw=black!50, dashed]
            (a) rectangle (c);
        \path (asr1.north west)+(-0.2,0.2) node (a) {};

    \end{pgfonlayer}



\end{tikzpicture}

        % \end{figure}
      \end{frame}

      \begin{frame}
        \frametitle{Automatic Doubling Experiments}
        \begin{itemize}
          \item Diagram
          \item Convergence Algs
          \item Doubling Schemas
        \end{itemize}
      \end{frame}

    \section{Experiment Design}
      \begin{frame}
        \frametitle{Experiments}
        \begin{itemize}
          \item Parameters Studied
          \item Cluster
        \end{itemize}
      \end{frame}

    \section{Results}
      \begin{frame}
        \frametitle{Results}
        \begin{itemize}
          \item BigOh Table
          \item Data mining
          \item Statistical Tests
        \end{itemize}
      \end{frame}

    \section{Conclusion}
      \begin{frame}
        \frametitle{Conclusion}
      \end{frame}

      % \begin{frame}
      %   \titlepage
      % \end{frame}

      \end{document}
